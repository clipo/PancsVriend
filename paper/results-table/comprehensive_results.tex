\documentclass[11pt]{article}
\usepackage[margin=0.8in, landscape]{geometry}
\usepackage{amsmath,amssymb,amsfonts}
\usepackage{booktabs}
\usepackage{float}
\usepackage{afterpage}

\begin{document}

\title{Segregation Metrics Analysis: Complete Results and Documentation}
\author{}
\date{\today}
\maketitle

\begin{abstract}
This document presents comprehensive segregation metrics comparing LLM agent behavior across five social contexts within the Schelling framework: baseline control (green/yellow), racial (race:wht/blk), ethnic (ethnicity), economic (income), and political contexts. The analysis employs the Pancs-Vriend framework with six segregation metrics and includes both individual metric comparisons and an overall consensus ordering derived from cross-metric agreement patterns.
\end{abstract}

\tableofcontents
\newpage

\section{Results Tables}

The following tables present segregation metric comparisons across different social contexts. Table~\ref{tab:segregation_metrics} shows the complete analysis including mechanical agents, while Table~\ref{tab:segregation_metrics_llm} focuses exclusively on LLM-based scenarios. Each table includes an overall consensus ordering synthesized from all six metrics, followed by individual metric comparisons with statistical significance indicators.

\subsection{Complete Analysis (All Scenarios)}

\begin{table}[htbp]
\centering
\caption{Segregation Metrics Comparison Across Scenarios}
\label{tab:segregation_metrics}
\vspace{1.5em}
\begin{align*}
\text{overall:} & \quad \begin{pmatrix} \text{income} \end{pmatrix} & \leq & \begin{pmatrix} \text{green/yellow} \end{pmatrix} & \leq & \begin{pmatrix} \text{llm\_baseline} \end{pmatrix} & < & \begin{pmatrix} \text{race:wht/blk} \end{pmatrix} & \leq & \begin{pmatrix} \text{ethnicity} \end{pmatrix} & \leq & \begin{pmatrix} \text{mechanical} \end{pmatrix} & < & \begin{pmatrix} \text{political} \end{pmatrix} \\\\[2.5em]
\hline \\\\[0.5em]
\text{-clusters:} & \quad \begin{pmatrix} \text{green/yellow} \\ \text{income} \end{pmatrix} & <^{***} & \begin{pmatrix} \text{llm\_baseline} \end{pmatrix} & <^{***} & \begin{pmatrix} \text{race:wht/blk} \end{pmatrix} & <^{***} & \begin{pmatrix} \text{ethnicity} \end{pmatrix} & <^{***} & \begin{pmatrix} \text{mechanical} \\ \text{political} \end{pmatrix} \\\\[1em]
\text{distance:} & \quad \begin{pmatrix} \text{green/yellow} \\ \text{income} \\ \text{llm\_baseline} \end{pmatrix} & <^{***} & \begin{pmatrix} \text{ethnicity} \\ \text{mechanical} \\ \text{race:wht/blk} \end{pmatrix} & <^{***} & \begin{pmatrix} \text{political} \end{pmatrix} \\\\[2em]
\text{ghetto\_rate:} & \quad \begin{pmatrix} \text{green/yellow} \\ \text{income} \end{pmatrix} & <^{*} & \begin{pmatrix} \text{llm\_baseline} \end{pmatrix} & <^{***} & \begin{pmatrix} \text{ethnicity} \\ \text{race:wht/blk} \end{pmatrix} & <^{**} & \begin{pmatrix} \text{mechanical} \end{pmatrix} & <^{***} & \begin{pmatrix} \text{political} \end{pmatrix} \\\\[1em]
\text{mix\_deviation:} & \quad \begin{pmatrix} \text{green/yellow} \end{pmatrix} & <^{**} & \begin{pmatrix} \text{income} \\ \text{llm\_baseline} \end{pmatrix} & <^{***} & \begin{pmatrix} \text{ethnicity} \\ \text{mechanical} \\ \text{race:wht/blk} \end{pmatrix} & <^{***} & \begin{pmatrix} \text{political} \end{pmatrix} \\\\[2em]
\text{share:} & \quad \begin{pmatrix} \text{green/yellow} \\ \text{income} \end{pmatrix} & <^{*} & \begin{pmatrix} \text{llm\_baseline} \end{pmatrix} & <^{***} & \begin{pmatrix} \text{ethnicity} \\ \text{race:wht/blk} \end{pmatrix} & <^{***} & \begin{pmatrix} \text{mechanical} \end{pmatrix} & <^{***} & \begin{pmatrix} \text{political} \end{pmatrix} \\\\[1em]
\text{-switch\_rate:} & \quad \begin{pmatrix} \text{green/yellow} \\ \text{income} \\ \text{llm\_baseline} \end{pmatrix} & <^{***} & \begin{pmatrix} \text{ethnicity} \\ \text{mechanical} \\ \text{race:wht/blk} \end{pmatrix} & <^{***} & \begin{pmatrix} \text{political} \end{pmatrix}
\end{align*}
\vspace{2em}

\small{Six metrics where more of the metric is associated with higher segregation.\\
Higher clusters and switch\_rate are associated with \textit{less} segregation, so -clusters and -switch\_rate are presented here for easier comparison.\\
Note: There is general agreement with the order.}
\end{table}

\newpage

\subsection{LLM-Only Analysis}

\begin{table}[htbp]
\centering
\caption{Segregation Metrics Comparison Across LLM Scenarios}
\label{tab:segregation_metrics_llm}
\vspace{1.5em}
\begin{align*}
\text{overall:} & \quad \begin{pmatrix} \text{income} \end{pmatrix} & \leq & \begin{pmatrix} \text{green/yellow} \end{pmatrix} & \leq & \begin{pmatrix} \text{llm\_baseline} \end{pmatrix} & < & \begin{pmatrix} \text{race:wht/blk} \end{pmatrix} & \leq & \begin{pmatrix} \text{ethnicity} \end{pmatrix} & < & \begin{pmatrix} \text{political} \end{pmatrix} \\\\[2.5em]
\hline \\\\[0.5em]
\text{-clusters:} & \quad \begin{pmatrix} \text{green/yellow} \\ \text{income} \end{pmatrix} & <^{***} & \begin{pmatrix} \text{llm\_baseline} \end{pmatrix} & <^{***} & \begin{pmatrix} \text{race:wht/blk} \end{pmatrix} & <^{***} & \begin{pmatrix} \text{ethnicity} \end{pmatrix} & <^{***} & \begin{pmatrix} \text{political} \end{pmatrix} \\\\[1em]
\text{distance:} & \quad \begin{pmatrix} \text{green/yellow} \\ \text{income} \\ \text{llm\_baseline} \end{pmatrix} & <^{***} & \begin{pmatrix} \text{ethnicity} \\ \text{race:wht/blk} \end{pmatrix} & <^{***} & \begin{pmatrix} \text{political} \end{pmatrix} \\\\[2em]
\text{ghetto\_rate:} & \quad \begin{pmatrix} \text{green/yellow} \\ \text{income} \end{pmatrix} & <^{*} & \begin{pmatrix} \text{llm\_baseline} \end{pmatrix} & <^{***} & \begin{pmatrix} \text{ethnicity} \\ \text{race:wht/blk} \end{pmatrix} & <^{**} & \begin{pmatrix} \text{political} \end{pmatrix} \\\\[1em]
\text{mix\_deviation:} & \quad \begin{pmatrix} \text{green/yellow} \end{pmatrix} & <^{**} & \begin{pmatrix} \text{income} \\ \text{llm\_baseline} \end{pmatrix} & <^{***} & \begin{pmatrix} \text{ethnicity} \\ \text{race:wht/blk} \end{pmatrix} & <^{***} & \begin{pmatrix} \text{political} \end{pmatrix} \\\\[1em]
\text{share:} & \quad \begin{pmatrix} \text{green/yellow} \\ \text{income} \end{pmatrix} & <^{*} & \begin{pmatrix} \text{llm\_baseline} \end{pmatrix} & <^{***} & \begin{pmatrix} \text{ethnicity} \\ \text{race:wht/blk} \end{pmatrix} & <^{***} & \begin{pmatrix} \text{political} \end{pmatrix} \\\\[1em]
\text{-switch\_rate:} & \quad \begin{pmatrix} \text{green/yellow} \\ \text{income} \\ \text{llm\_baseline} \end{pmatrix} & <^{***} & \begin{pmatrix} \text{ethnicity} \\ \text{race:wht/blk} \end{pmatrix} & <^{***} & \begin{pmatrix} \text{political} \end{pmatrix}
\end{align*}
\vspace{2em}

\small{Six metrics where more of the metric is associated with higher segregation.\\
Higher clusters and switch\_rate are associated with \textit{less} segregation, so -clusters and -switch\_rate are presented here for easier comparison.\\
Note: There is general agreement with the order.}
\end{table}

\newpage

\section{Documentation}

\section*{Legend and Glossary}

\subsection*{Social Context Scenarios}

The segregation metrics compare agent behavior across five distinct social contexts within the Schelling framework:

\begin{description}
\item[\textbf{green/yellow (Baseline Control)}] Generic ``red vs blue'' teams without social connotations, serving as a neutral baseline
\item[\textbf{race:wht/blk (Racial Context)}] ``White middle-class families'' vs ``Black families'' - captures residential preferences based on racial identity
\item[\textbf{ethnicity (Ethnic Context)}] ``Asian American families'' vs ``Hispanic/Latino families'' - examines ethnicity-based residential patterns  
\item[\textbf{income (Economic Context)}] ``High-income professionals'' vs ``Working-class families'' - tests segregation based on economic class
\item[\textbf{political (Political Context)}] ``Liberal households'' vs ``Conservative households'' - measures political polarization in residential choices
\item[\textbf{llm\_baseline}] LLM agents using the baseline control scenario
\item[\textbf{mechanical}] Traditional utility-maximizing agents with threshold-based decision rules (non-LLM baseline)
\end{description}

\subsection*{Segregation Metrics}

The analysis employs the Pancs-Vriend framework with six complementary metrics designed specifically for grid-based segregation models:

\begin{description}
\item[\textbf{Share}] Proportion of same-type neighbor pairs (0.5 = perfect integration, 1.0 = complete segregation). Captures global segregation level.

\item[\textbf{Clusters}] Number of spatially contiguous same-type regions. Fewer clusters indicate more consolidated ethnic enclaves. \textit{Note: Presented as -clusters since higher cluster counts indicate less segregation.}

\item[\textbf{Distance}] Average Manhattan distance to nearest different-type agent. Higher values indicate greater spatial separation between groups.

\item[\textbf{Ghetto Rate}] Count of agents with zero different-type neighbors. Captures extreme isolation and ``ghettoization'' - the most severe form of segregation.

\item[\textbf{Mix Deviation}] Average deviation from 50-50 local integration. Measures segregation at the individual neighborhood level.

\item[\textbf{Switch Rate}] Frequency of type changes along agent borders. Higher values indicate more jagged, intermixed boundaries. \textit{Note: Presented as -switch\_rate since higher switch rates indicate less segregation.}
\end{description}

\subsection*{Statistical Significance Notation}

The inequality symbols indicate the strength of statistical relationships between scenarios:

\begin{description}
\item[\textbf{<}] Strong consensus ($\geq$5 out of 6 metrics agree) - highly significant difference
\item[\textbf{$\leq$}] Weak consensus ($\geq$1 metric agrees, no disagreement) - generally supported relationship  
\item[\textbf{$\leq$\textsuperscript{?}}] Mixed evidence (both agreements and disagreements) - uncertain relationship
\item[\textbf{=}] Perfect agreement (all 6 metrics agree) - equivalent segregation levels
\end{description}

\subsection*{Significance Levels}

Statistical significance is denoted by asterisk superscripts:

\begin{description}
\item[\textbf{*}] p < 0.05 (statistically significant)
\item[\textbf{**}] p < 0.01 (highly significant)  
\item[\textbf{***}] p < 0.001 (extremely significant)
\end{description}

\subsection*{Interpretation}

\textbf{Overall Ordering:} The consensus ranking synthesizes evidence across all six metrics to provide a single ordering from lowest to highest segregation levels.

\textbf{Individual Metrics:} Each metric captures a different dimension of segregation - from global patterns (Share) to local isolation (Ghetto Rate) to spatial structure (Clusters, Distance). 

\textbf{Key Finding:} Political contexts produce the most extreme segregation (ghetto rate: 61.6, segregation share: 0.928), while economic contexts show minimal clustering (ghetto rate: 5.0, share: 0.543) - a 12.3× difference in ghetto formation based solely on social framing.

\section{Methodology Notes}

\subsection{Consensus Ordering Calculation}

The overall ordering represents a synthesis of evidence across all six segregation metrics. For each adjacent pair of scenarios in the ordering:

\begin{itemize}
\item We count how many metrics support each possible relationship (less than, greater than, or equal)
\item Consensus rules are applied:
  \begin{itemize}
  \item Strong consensus ($<$): $\geq$5 out of 6 metrics agree on direction
  \item Weak consensus ($\leq$): $\geq$1 metric agrees with no disagreement  
  \item Mixed evidence ($\leq^?$): Both agreement and disagreement present
  \item Perfect agreement ($=$): All 6 metrics show identical values
  \end{itemize}
\end{itemize}

\subsection{Key Research Implications}

The results demonstrate that LLM agents produce dramatically different segregation patterns based solely on social context framing:

\begin{itemize}
\item \textbf{Political contexts} show extreme segregation (ghetto rate: 61.6, share: 0.928)
\item \textbf{Economic contexts} show minimal clustering (ghetto rate: 5.0, share: 0.543)  
\item This represents a \textbf{12.3× difference} in ghetto formation based purely on social framing
\item All scenarios differ significantly from baseline (p < 0.001)
\end{itemize}

These findings suggest that LLMs can capture culturally-embedded preferences and biases, producing segregation dynamics that vary realistically with social context.

\section{References}

For complete methodological details and theoretical background, see:

\textit{Social Context Matters: How Large Language Model Agents Reproduce Real-World Segregation Patterns in the Schelling Model} (Pancs \& Vriend, submitted to Journal of Economic Interaction and Coordination).

\end{document}